\section*{ВВЕДЕНИЕ} % Секция без номера
\phantomsection
\addcontentsline{toc}{section}{ВВЕДЕНИЕ} % Добавить в содержание

В современном мире программисту приходится сталкиваться с огромным количеством информации, которую необходимо сохранить. Хранение информации – это ее запись во вспомогательные запоминающие устройства на различных носителях для последующего использования. Хранение является одной из основных операций, осуществляемых над информацией, и главным способом обеспечения ее доступности в течение определенного промежутка времени.

Данная курсовая работа предполагает решение следующих задач:

\begin{enumerate}
    \item Изучить, что такое информационные системы, принципы их создания и работы, объяснить их необходимость в жизни человека и описать сферы применения информационной системы <<Товары>>
    \item Изучить возможности и характеристики при создании приложения для работы с информационной системой <<Товары>>
    \item Спроектировать систему и создать приложение с данными на тему <<Товары>>.
\end{enumerate}

Информационная система (ИС) — система, предназначенная для хранения, поиска и обработки информации, и соответствующие организационные ресурсы (человеческие, технические, финансовые и т. д.), которые обеспечивают и распространяют информацию.

ИС предназначена для своевременного обеспечения надлежащих людей надлежащей информацией, то есть для удовлетворения конкретных информационных потребностей в рамках определённой предметной области, при этом результатом функционирования информационных систем является информационная продукция — документы, информационные массивы, базы данных и информационные услуги.

Первоначально для хранения информации на ЭВМ применялись локальные массивы (или файлы), при этом для каждой из решаемых функциональных задач создавались собственные файлы исходной и результатной информации. Это приводило к значительному дублированию данных, за счёт чего использовалось больше памяти вычислительной машины, а также усложнялось обновление хранимой информации.

С целью решения вышеописанных проблем были созданы своего рода электронные хранилища данных - базы данных. Базы данных - это часть информационных систем.

База данных представляет собой определенным образом структурированную совокупность данных, совместно хранящихся и обрабатывающихся в соответствии с некоторыми правилами. Как правило, БД моделирует некоторую предметную область или ее фрагмент. Очень часто в качестве постоянного хранилища информации баз данных выступают файлы.

Немаловажной является и взаимосвязь информации в базе данных: изменение одной строчки может привести к значительным изменениям других строк. Работать с данными таким образом гораздо проще и быстрее, чем если бы изменения касались только одного места в базе данных.

Помимо основной функции - хранения и систематизации огромного количества информации - они позволяют быстро обрабатывать клиентские запросы и выдавать актуальную информацию.

На сегодняшний день базы данных занимают одно из первых мест для многих организаций, которые для упрощения своей работы применяют компьютерные технологии.

Результатом разрабатываемой программы должно являться приложение, позволяющее пользователю взаимодействовать с данными о спортсменах футбольной команды при помощи пользовательского интерфейса.

\newpage
