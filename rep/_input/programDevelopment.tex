\newpage

\section{РАЗРАБОТКА ПРОГРАММЫ}

\subsection{Выбор средств программирования}

Сервер: LAMP - Linux Apache2 MySQL PHP.

Для установки сервера на Ubuntu подобных системах использовались команды:

\begin{verbatim}
sudo apt update

sudo apt install apache2 php libapache2-mod-php php-mysql mysql-server phpmyadmin
\end{verbatim}

После установки пакета "phpmyadmin" консоль предложит придумать пароль. Этот пароль будет пользователя под логином "phpmyadmin". Заходим на сраницу localhost/phpmyadmin под логином "phpmyadmin" и придуманым паролем - и видем, что мы не можем создавать базы данных. Для создания пользователя с такими привилегиями выполним скрипты:

\begin{verbatim}
sudo mysql -u root -p

CREATE USER 'admin'@'localhost' IDENTIFIED BY '111111';

GRANT ALL PRIVILEGES ON *.* TO 'admin'@'localhost' WITH GRAND OPTION;

FLUSH PRIVILEGES;
\end{verbatim}

Заходим на сраницу localhost/phpmyadmin под логином admin и паролем 111111 - и теперь можем создавать базу данных.

\hspace{0pt}\\

Задания выполнялись на языке программирования PHP. Верстка сайта велась по библиотеке Bootstrap. Для выполнения задачи на языке программирования PHP не потребовалось подключение библиотек. 

\hspace{0pt}

Среда разработки: Visual Studio Code.

\hspace{0pt}\\

ОС: Linux / Windows (при Windows можно использовать Open Server вместо LAMP)

\hspace{0pt}\\

В ходе написания проекта, проект был разбит на модули.

\hspace{0pt}\\

Использованы модули-сраницы: 

\begin{itemize}
    \item add
    \item show
\end{itemize}

\hspace{0pt}\\

Использованы модули-скрипты: 

\begin{itemize}
    \item form
\end{itemize}

\hspace{0pt}\\

Использованы многоразовые модули (для include): 

\begin{itemize}
    \item connect
    \item header
    \item menu
\end{itemize}

\newpage

\subsection{Разработка модулей}

\textbf{Модуль add}

\underline{Подключённые модули}:

\begin{enumerate}
    \item header
    \item menu
\end{enumerate}

\underline{Входные параметры}:

\begin{enumerate}
    \item Нет
\end{enumerate}

\underline{Назначение}: страница, которая имеет поля. После отправки, которые будет добавлены в базу MySQL.

\underline{Возвращаемые данные}: добавлены поля в базу данных MySQL.

\hspace{0pt}\\

% = = =

\textbf{Модуль show}

\underline{Подключённые модули}:

\begin{enumerate}
    \item header
    \item menu
    \item connect
\end{enumerate}

\underline{Входные параметры}:

\begin{enumerate}
    \item Нет
\end{enumerate}

\underline{Назначение}: страница, которая возвращает код HTML с удобочитаемой таблицой с даными.

\underline{Возвращаемые данные}: страница с отрисованной таблицей HTML.

\hspace{0pt}\\

% = = =

\textbf{Модуль form}

\underline{Подключённые модули}:

\begin{enumerate}
    \item connect
\end{enumerate}

\underline{Входные параметры}:

\begin{enumerate}
    \item model - модель (желаемый тип - строка)
    \item name - имя (желаемый тип - строка)
    \item onBox - количество в коробке (желаемый тип - не отрицательное целочисленное значение)
    \item weight - вес (желаемый тип - число с плавающей точкой)
    \item m3 - объем (желаемый тип - число с плавающей точкой)     
    \item series - серия (желаемый тип - строка) 
\end{enumerate}

\underline{Назначение}: скрипт, который принимает входный параметры по средствам POST запросов. Использует входные параметры для добавления в базу данных.

\underline{Возвращаемые данные}: данные появяться в базе MySQL. Сраница перенаправаться в корень.

\hspace{0pt}\\

% = = =

\textbf{Модуль connect}

\underline{Входные параметры}:

\begin{enumerate}
    \item site - сайт (желаемы тип - строка, например,  "localhost")
    \item userDB - логин в базе данных (желаемый тип - строка, например, "admin")
    \item passwordDB - пароль от базы данных (желаемый тип строка, например, "111111")
    \item database - имя базы данных (желаемый тип - строка, например, "productsdb")
    \item table - имя таблицы (желаемый тип - строка, например, "productstable")
\end{enumerate}

\underline{Назначение}: скрипт, который под входными данными подключится к базе данных.

\underline{Возвращаемые данные}: удачное/неудачное подключение к базе данных

\hspace{0pt}\\

% = = =

\textbf{Модуль header}

\underline{Входные параметры}:

\begin{enumerate}
    \item title - название страницы (желаемый тип - строка)
\end{enumerate}

\underline{Назначение}: страница с кодом HTML c заполненым тегом head.

\underline{Возвращаемые данные}: вставленый код HTML.

\hspace{0pt}\\

% = = =

\newpage

\textbf{Модуль menu}

\underline{Входные параметры}:

\begin{enumerate}
    \item нет
\end{enumerate}

\underline{Назначение}: страница с кодом HTML c заполненым меню.

\underline{Возвращаемые данные}: вставленый код HTML.

\hspace{0pt}\\

% = = =

\newpage