\newpage
\section{РАЗРАБОТКА АЛГОРИТМОВ}

\subsubsection*{Алгоритм подключения к базе данных}

\underline{Исходные данные}:
\begin{enumerate}
    \item имя сервера
    \item логин от phpmyadmin
    \item пароль от phpmyadmin
    \item имя базы данных
\end{enumerate}

\underline{Алгоритм}:
\begin{enumerate}
    \item производим SQL запрос базе данных
    \item Если запрос не удачный, то вывести ошибку и закончить выполнение скрипта
\end{enumerate}

\underline{Выходные данные}:
\begin{enumerate}
    \item сформированно подключение к базе данных
\end{enumerate}

% = = =

\subsubsection*{Алгоритм добавление нового элемента}

\underline{Исходные данные}:
\begin{enumerate}
    \item форма с полями, которые получены методом GET
    \item база MySQL
\end{enumerate}

\underline{Алгоритм}:
\begin{enumerate}
    \item взятие полей из формы методом GET
    \item сохраниние полей в таблицу с помощью SQL запроса в базу данных phpmyadmin
\end{enumerate}

\underline{Выходные данные}:
\begin{enumerate}
    \item база данных с добавленным элементом
    \item перенаправление на страницу вывода таблицы
\end{enumerate}

% = = =

\subsubsection*{Алгоритм просмотр элементов}

\underline{Исходные данные}: 
\begin{enumerate}
    \item база MySQL
\end{enumerate}

\underline{Алгоритм}:
\begin{enumerate}
    \item вывод шапки таблицы
    \item получение массива элементов из базы данных
    \item приминение фильтров (поиска)
    \item применение сортировки по полю
    \item цикличный вывод элементов
\end{enumerate}

\underline{Выходные данные}:
\begin{enumerate}
    \item сформированная HTML страница
\end{enumerate}

% = = =

\subsubsection*{Алгоритм удаления элемента}

\underline{Исходные данные}:
\begin{enumerate}
    \item HTML таблица
\end{enumerate}

\underline{Алгоритм}:
\begin{enumerate}
    \item с помощью таблицы зная элемент, выбираем его, нажав например на мусорку, которая дает GET запрос с ID
    \item получаем ID через GET запрос
    \item выполняем скрипт подключения к базе данных
    \item выполняем SQL запрос по удалению по полю ID
\end{enumerate}

\underline{Выходные данные}:
\begin{enumerate}
    \item база данных с удаленным элементом
    \item перенаправление на эту же страницы вывода таблицы
\end{enumerate}

% = = =

\subsubsection*{Алгоритм удаления элемента}

\underline{Исходные данные}:
\begin{enumerate}
    \item HTML таблица
\end{enumerate}

\underline{Алгоритм}:
\begin{enumerate}
    \item с помощью таблицы зная элемент, выбираем его, нажав например на карандаш, которая дает GET запрос с ID
    \item получаем ID через GET запрос
    \item перенаправляемся на страницу с редактированием полей
    \item выполняем скрипт подключения к базе данных
    \item выполняем SQL запрос замене полей по ID
\end{enumerate}

\underline{Выходные данные}:
\begin{enumerate}
    \item база данных с обновленным элементом
    \item перенаправление на страницу вывода таблицы
\end{enumerate}

% = = =

\subsubsection*{Алгоритм создания таблицы}

\underline{Исходные данные}:
\begin{enumerate}
    \item удачное подключение к базе данных
\end{enumerate}

\underline{Алгоритм}:
\begin{enumerate}
    \item производим SQL запрос базе данных для создания таблицы
    \item Если запрос не удачный, то вывести ошибку
\end{enumerate}

\underline{Выходные данные}:
\begin{enumerate}
    \item в базе данных появилась таблица
\end{enumerate}

% = = =

\subsubsection*{Алгоритм удаления таблицы}

\underline{Исходные данные}:
\begin{enumerate}
    \item удачное подключение к базе данных
\end{enumerate}

\underline{Алгоритм}:
\begin{enumerate}
    \item производим SQL запрос базе данных для удаления таблицы
    \item Если запрос не удачный, то вывести ошибку
\end{enumerate}

\underline{Выходные данные}:
\begin{enumerate}
    \item в базе данных удалена таблица
\end{enumerate}

\newpage